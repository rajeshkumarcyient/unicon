\subsection*{Preface to Part II}
\addcontentsline{toc}{section}{Preface to Part II}

\bigskip

There are many optimizations that can be applied while translating
Icon programs. These optimizations and the analyses needed to apply
them are of interest for two reasons. First, Icon's unique combination
of characteristics requires developing new techniques for implementing
them. Second, these optimizations are useful in variety of languages
and Icon can be used as a medium for extending the state of the art.

Many of these optimizations require detailed control of the generated
code. Previous production implementations of the Icon programming
language have been interpreters. The virtual machine code of an
interpreter is seldom flexible enough to accommodate these
optimizations and modifying the virtual machine to add the flexibility
destroys the simplicity that justified using an interpreter in the
first place. These optimizations can only reasonably be implemented in
a compiler. In order to explore these optimizations for Icon programs,
a compiler was developed. This part of the compendium describes the
compiler and the optimizations it employs. It also describes a
run-time system designed to support the analyses and optimizations.

Icon variables are untyped. The compiler contains a type inferencing
system that determines what values variables and expression may take
on during program execution. This system is effective in the presence
of values with pointer semantics and of assignments to components of
data structures.

The compiler stores intermediate results in temporary variables rather
than on a stack. A simple and efficient algorithm was developed for
determining the lifetimes of intermediate results in the presence of
goal-directed evaluation. This allows an efficient allocation of
temporary variables to intermediate results.

The compiler uses information from type inferencing and liveness
analysis to simplify generated code. Performance measurements on a
variety of Icon programs show these optimizations to be effective.

The optimizing compiler for Icon was developed by Ken Walker as part
of his Ph.D. research, and this part of the Icon/Unicon Compendium is
essentially a reprint of his dissertation, which also appeared as
University of Arizona CS TR 91-16. Along with his consent, Ken kindly
provided the original groff sources to his dissertation. Any
typographical and formatting errors that remain are the fault of the
editor.
