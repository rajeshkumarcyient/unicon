\chapter{Preface to the Second Edition}

This book will raise your level of skill at computer programming,
regardless of whether you are presently a novice or expert. The field
of programming languages is still in its infancy, and dramatic advances
will be made every decade or two until mankind has had enough time to
think about the problems and principles that go into this exciting area
of computing. The Unicon language described in this book is such an
advance, incorporating many elegant ideas not yet found in most
contemporary languages.

Unicon is an object-oriented, goal-directed programming language based
on the Icon programming language. Unicon can be pronounced however you
wish; we pronounce it variably depending on mood, whim, or situation;
the most frequent pronunciation rhymes with ``lexicon''.

For Icon programmers this work serves as a ``companion book'' that documents
material such as the Icon Program Library, a valuable resource that is
underutilized.  Don't be surprised by language changes: the book presents many
new facilities that were added to Icon to make Unicon and gives examples from
new application areas to which Unicon is well suited. For people new to Icon
and Unicon, this book is an exciting guide to a powerful language.

It is with sweet irony that we call this book the 2\textsuperscript{nd}
Edition, since the first edition was never formally published but
instead existed solely as an online document, although laser-printed
hard copies could be requested. A lot has happened to Unicon since the
first edition of this book, which culminated in 2004. This
{\textquotedblleft}2\textsuperscript{nd} Edition{\textquotedblright}
catches readers up with things like concurrent threads and vastly
improved 3D graphics facilities. Along the way, the games chapter and
parts of the internet programming chapter got spun off into a separate
work, the so-called \textit{Manual of Puissant Skill at Game
Programming}. 

\section*{Organization of This Book}

This book consists of four parts. The first part, Chapters 1-8, presents
the core of the Unicon language, much of which comes from Icon. These
early chapters start with simple expressions, progress through data
structures and string processing, and include advanced programming
topics and the input/output capabilities of Unicon's
portable system interface. Part two, in Chapters 9-12,
describes object-oriented development as a whole and presents
Unicon's object-oriented facilities in the context of
object-oriented design. Object-oriented programming in Unicon
corresponds closely to object-oriented design diagrams in the Unified
Modeling Language, UML. Some of the most interesting parts of the book
are in part three; Chapters 13-18 provide example programs that use
Unicon in a wide range of application areas. Part four consists of
essential reference material presented in several Appendixes.

\section*{Acknowledgments}

Thanks to the Icon Project for creating a most excellent language.
Thanks especially to those unsung heroes, the university students and
Internet volunteers who implemented the language and its program
library over a period of many years. Icon contributors can be divided
into epochs. In the epoch leading up to the first edition of this
book, we were inspired by contributions from Gregg Townsend, Darren
Merrill, Mary Cameron, Jon Lipp, Anthony Jones, Richard Hatch,
Federico Balbi, Todd Proebsting, Steve Lumos and Naomi Martinez.  In
the epoch since the first edition of this book, the Unicon Project
owes a debt of gratitude to Ziad al Sharif, Hani bani Salameh, Jafar
Al Gharaibeh, Mike Wilder, and Sudarshan Gaikaiwari.

The most impressive contributors are those whose influence on Icon has
spanned across epochs, such as Ralph Griswold, Steve Wampler, Bob
Alexander, Ken Walker, Phillip Thomas, and Kostas Oikonomou. We revere
you folks! Steve Wampler deserves extra thanks for serving as the
technical reviewer for the first edition of this book.  Phillip Thomas
and Kostas Oikonomou have provided extensive support and assistance
that goes way beyond the call of duty; in many ways this is their book.

This manuscript received critical improvements and corrections from
many additional technical reviewers, including, David
A.  Gamey, Craig S. Kaplan, David Feustel, David Slate,
Frank Lhota, Art Eschenlauer, Wendell Turner, Dennis
Darland, and Nolan Clayton.

The authors wish to acknowledge generous support from the National
Library of Medicine and AT\&T Bell Labs Research.
This work was also supported in part by the National
Science Foundation under grants CDA-9633299, EIA-0220590 and
EIA-9810732, and the Alliance for Minority Participation.

Clinton Jeffery

Shamim Mohamed

Jafar al Gharaibeh

Ray Pereda

Robert Parlett
